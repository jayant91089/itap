% generated by GAPDoc2LaTeX from XML source (Frank Luebeck)
\documentclass[a4paper,11pt]{report}

\usepackage{a4wide}
\sloppy
\pagestyle{myheadings}
\usepackage{amssymb}
\usepackage[utf8]{inputenc}
\usepackage{makeidx}
\makeindex
\usepackage{color}
\definecolor{FireBrick}{rgb}{0.5812,0.0074,0.0083}
\definecolor{RoyalBlue}{rgb}{0.0236,0.0894,0.6179}
\definecolor{RoyalGreen}{rgb}{0.0236,0.6179,0.0894}
\definecolor{RoyalRed}{rgb}{0.6179,0.0236,0.0894}
\definecolor{LightBlue}{rgb}{0.8544,0.9511,1.0000}
\definecolor{Black}{rgb}{0.0,0.0,0.0}

\definecolor{linkColor}{rgb}{0.0,0.0,0.554}
\definecolor{citeColor}{rgb}{0.0,0.0,0.554}
\definecolor{fileColor}{rgb}{0.0,0.0,0.554}
\definecolor{urlColor}{rgb}{0.0,0.0,0.554}
\definecolor{promptColor}{rgb}{0.0,0.0,0.589}
\definecolor{brkpromptColor}{rgb}{0.589,0.0,0.0}
\definecolor{gapinputColor}{rgb}{0.589,0.0,0.0}
\definecolor{gapoutputColor}{rgb}{0.0,0.0,0.0}

%%  for a long time these were red and blue by default,
%%  now black, but keep variables to overwrite
\definecolor{FuncColor}{rgb}{0.0,0.0,0.0}
%% strange name because of pdflatex bug:
\definecolor{Chapter }{rgb}{0.0,0.0,0.0}
\definecolor{DarkOlive}{rgb}{0.1047,0.2412,0.0064}


\usepackage{fancyvrb}

\usepackage{mathptmx,helvet}
\usepackage[T1]{fontenc}
\usepackage{textcomp}


\usepackage[
            pdftex=true,
            bookmarks=true,        
            a4paper=true,
            pdftitle={Written with GAPDoc},
            pdfcreator={LaTeX with hyperref package / GAPDoc},
            colorlinks=true,
            backref=page,
            breaklinks=true,
            linkcolor=linkColor,
            citecolor=citeColor,
            filecolor=fileColor,
            urlcolor=urlColor,
            pdfpagemode={UseNone}, 
           ]{hyperref}

\newcommand{\maintitlesize}{\fontsize{50}{55}\selectfont}

% write page numbers to a .pnr log file for online help
\newwrite\pagenrlog
\immediate\openout\pagenrlog =\jobname.pnr
\immediate\write\pagenrlog{PAGENRS := [}
\newcommand{\logpage}[1]{\protect\write\pagenrlog{#1, \thepage,}}
%% were never documented, give conflicts with some additional packages

\newcommand{\GAP}{\textsf{GAP}}

%% nicer description environments, allows long labels
\usepackage{enumitem}
\setdescription{style=nextline}

%% depth of toc
\setcounter{tocdepth}{1}





%% command for ColorPrompt style examples
\newcommand{\gapprompt}[1]{\color{promptColor}{\bfseries #1}}
\newcommand{\gapbrkprompt}[1]{\color{brkpromptColor}{\bfseries #1}}
\newcommand{\gapinput}[1]{\color{gapinputColor}{#1}}


\begin{document}

\logpage{[ 0, 0, 0 ]}
\begin{titlepage}
\mbox{}\vfill

\begin{center}{\maintitlesize \textbf{ itap \mbox{}}}\\
\vfill

\hypersetup{pdftitle= itap }
\markright{\scriptsize \mbox{}\hfill  itap  \hfill\mbox{}}
{\Huge \textbf{ Information Theoretic Achievability Prover \mbox{}}}\\
\vfill

{\Huge  1.0 \mbox{}}\\[1cm]
{ 25/08/2015 \mbox{}}\\[1cm]
\mbox{}\\[2cm]
{\Large \textbf{ Jayant Apte\\
    \mbox{}}}\\
{\Large \textbf{ John Walsh\\
    \mbox{}}}\\
\hypersetup{pdfauthor= Jayant Apte\\
    ;  John Walsh\\
    }
\end{center}\vfill

\mbox{}\\
{\mbox{}\\
\small \noindent \textbf{ Jayant Apte\\
    }  Email: \href{mailto://jayant91089@gmail.com} {\texttt{jayant91089@gmail.com}}\\
  Homepage: \href{https://sites.google.com/site/jayantapteshomepage/} {\texttt{https://sites.google.com/site/jayantapteshomepage/}}\\
  Address: \begin{minipage}[t]{8cm}\noindent
 Department of Electrical and Computer Engineering\\
 Drexel University\\
 Philadelphia, PA 19104\\
 \end{minipage}
}\\
{\mbox{}\\
\small \noindent \textbf{ John Walsh\\
    }  Email: \href{mailto://jwalsh@coe.drexel.edu} {\texttt{jwalsh@coe.drexel.edu}}\\
  Homepage: \href{http://www.ece.drexel.edu/walsh/web/} {\texttt{http://www.ece.drexel.edu/walsh/web/}}\\
  Address: \begin{minipage}[t]{8cm}\noindent
 Department of Electrical and Computer Engineering\\
 Drexel University\\
 Philadelphia, PA 19104\\
 \end{minipage}
}\\
\end{titlepage}

\newpage\setcounter{page}{2}
\newpage

\def\contentsname{Contents\logpage{[ 0, 0, 1 ]}}

\tableofcontents
\newpage

 \index{\textsf{itap}}     
\chapter{\textcolor{Chapter }{Introduction}}\label{Chapter_Introduction}
\logpage{[ 1, 0, 0 ]}
\hyperdef{L}{X7DFB63A97E67C0A1}{}
{
  ITAP stands for Information Theoretic Achievability Prover. So basically, it
is intended for coming up with achievability proofs using a computer. As of
now, it supports the following: 
\begin{itemize}
\item  Achievability proofs of multi-source network coding using vector-linear codes:
answer questions like 'Is this rate vector achievable using a vector linear
code over the given finite field?'. If the answer is `yes` it also returns the
vector linear code it found as a certificate of achievability. Otherwise, it
just says 'no'. 
\item  Achievability proofs with multi-linear secret sharing schemes: answer
questions 'Is there a multi-linear secret sharing scheme over $GF(q)$ for this access structure?' 
\item  Representability test for integer polymatroids over a given finite field:
answer questions like `Is the integer polymatroid associated with this rank
vector linear over $GF(q)$? 
\end{itemize}
 All three questions above are very similar, in that, we are looking for a
linear representation of an integer polymatroid satisfying certain properties
(satisfying network coding constraints, access structure and having a
specified rank vecor resp.) In the most general form an achievability prover
should be able to tell if there exists a joint distribution satisfying certain
constraints on entropy function which remains a fundamental open problem. ITAP
tries answering this in a more restricted sense, i.e. with vector linear
codes. The algorithm underlying $\texttt{itap}$ is called Leiterspiel or the algorithm of snakes and ladders. See \cite{betten2006error} for details. }

   
\chapter{\textcolor{Chapter }{Installation}}\label{Chapter_Installation}
\logpage{[ 2, 0, 0 ]}
\hyperdef{L}{X8360C04082558A12}{}
{
  To get the latest version of this GAP 4 package download one of the archive
files itap-x.x.zip, itap-x.x.tar.gz, itap-x.x.zoo, itap-x.x.tar.bz2 from \href{https://github.com/jayant91089/itap} {\texttt{https://github.com/jayant91089/itap}} or \href{http://www.ece.drexel.edu/walsh/aspitrg/software.html} {\texttt{http://www.ece.drexel.edu/walsh/aspitrg/software.html}} and unpack it using 
\[\texttt{gunzip itap-x.x.zip}\]
 or 
\[\texttt{gunzip itap-x.x.tar.gz}\]
 respectively and so on. Do this preferably (but not necessarily) inside the $\texttt{pkg}$ subdirectory of your GAP 4 installation. It creates a subdirectory called $\texttt{itap}$. This completes the installation of the package. One can then start using $\texttt{itap}$ by invoking 
\[\texttt{gap>}\]
 
\begin{Verbatim}[fontsize=\small,frame=single,label=Code]
  LoadPackage( "itap");
\end{Verbatim}
 from within GAP. $\textbf{Optional Dependencies}$: $\texttt{itap}$ can optionally use gap package $\texttt{FinInG}$. $\texttt{FinInG}$ provides allows internal functions to use projective semilinear group $P\Gamma L(k,q)$, which is the group of all collineations of the vector space $GF(q)^k$. See \href{http://cage.ugent.be/fining/} {\texttt{http://cage.ugent.be/fining/}} for information on how to obtain and install this package. $\texttt{LoadPackage( "itap")}$ automatically checks if $\texttt{FinInG}$ is available and loads it in case it is available. If $\texttt{FinInG}$ is absent, a light version of $\texttt{itap}$ is loaded, where the internal functions default to using projective general
linear group $PGL(k,q)\leq P\Gamma L(k,q)$. Note that this distinction is irrelevent if $q$ is a prime number as $PGL(k,q)\cong P\Gamma L(k,q)$. }

   
\chapter{\textcolor{Chapter }{Usage}}\label{Chapter_Usage}
\logpage{[ 3, 0, 0 ]}
\hyperdef{L}{X86A9B6F87E619FFF}{}
{
  
\section{\textcolor{Chapter }{Available functions}}\label{Chapter_Usage_Section_Available_functions}
\logpage{[ 3, 1, 0 ]}
\hyperdef{L}{X835D65A88292737E}{}
{
  In this section we shall look at the functions provided by $\texttt{itap}$. 

\subsection{\textcolor{Chapter }{proverep}}
\logpage{[ 3, 1, 1 ]}\nobreak
\hyperdef{L}{X7F1EE7A57984E68C}{}
{\noindent\textcolor{FuncColor}{$\triangleright$\ \ \texttt{proverep({\mdseries\slshape rankvec, nvars, F, optargs})\index{proverep@\texttt{proverep}}
\label{proverep}
}\hfill{\scriptsize (function)}}\\
\textbf{\indent Returns:\ }
A list 



 This function checks if there is a linear representation of an integer
polymatroid rank vector. It accepts following arguments: 
\begin{itemize}
\item  $\texttt{rankvec}$ - A list of integers specifying a polymatroid rank vector 
\item  $\texttt{nvars}$ - Number of ground set elements 
\item  $\texttt{F}$ - The finite field over which we wish to find a linear representation. It can
be defined by invoking the following in GAP: 
\end{itemize}
 
\begin{Verbatim}[fontsize=\small,frame=single,label=Code]
  q:=4;;
  F:= GF(q);; # here q must be a prime power
\end{Verbatim}
 
\begin{itemize}
\item  $\texttt{optargs}$ is a list of optional arguments $\texttt{[lazy,..]}$ where 
\begin{itemize}
\item  $\texttt{lazy}$ disables lazy evaluation of transporter maps if set to $\texttt{false}$, which is enabled by default in GAP. 
\end{itemize}
 
\end{itemize}
 Returns a list $\texttt{[true,code]}$ if there exists such a representation and $\texttt{code}$ is the vector linear code over $GF(q)$ Returns a list $\texttt{[false]}$ otherwise, indicating that no such representation exists }

 

\subsection{\textcolor{Chapter }{proverate}}
\logpage{[ 3, 1, 2 ]}\nobreak
\hyperdef{L}{X7DB2926A78EC4E86}{}
{\noindent\textcolor{FuncColor}{$\triangleright$\ \ \texttt{proverate({\mdseries\slshape ncinstance, rvec, F, optargs})\index{proverate@\texttt{proverate}}
\label{proverate}
}\hfill{\scriptsize (function)}}\\
\textbf{\indent Returns:\ }
A list 



 This function checks if there is a vector linear code achieving the rate
vector $\texttt{rvec}$ for the network coding instance $\texttt{ncinstance}$. It accepts following arguments: 
\begin{itemize}
\item  $\texttt{ncinstance}$ is a list $\texttt{[cons,nsrc,nvars]}$ containing 3 elements: 
\begin{itemize}
\item  $\texttt{cons}$ is a list of network coding constraints. 
\item  $\texttt{nsrc}$ is the number of sources. 
\item  $\texttt{nvars}$ is the number of random variables associated with the network. 
\end{itemize}
 
\item  $\texttt{rvec}$ - A list of $\texttt{nvars}$ integers specifying a rate vector 
\item  $ \texttt{F}$ is the finite field over which we wish to find the vector linear code. It can
be defined by invoking the following in GAP: 
\end{itemize}
 
\begin{Verbatim}[fontsize=\small,frame=single,label=Code]
  q:=4;;
  F:= GF(q);; # here q must be a prime power
\end{Verbatim}
 
\begin{itemize}
\item  $\texttt{optargs}$ is a list of optional arguments $\texttt{[lazy,..]}$ where $\texttt{lazy}$ disables lazy evaluation of transporter maps if set to $\texttt{false}$, which is enabled by default. 
\end{itemize}
 Returns a list $\texttt{[true,code]}$ if there exists such a representation and $\texttt{code}$ is the vector linear code over $GF(q)$ Returns a list $\texttt{[false]}$ otherwise, indicating that no such code exists }

 $\textbf{NOTE:}$ Certain naming convensions are followed while defining network coding
instances. The source messages are labeled with $\texttt{[1...nsrc]}$ while rest of the messages are labeled $\texttt{[nsrc...nvars]}$. Furthermore, the list $\texttt{cons}$ includes all network constraints except source independence. This constraint
is implied with the labeling i.e. $\texttt{itap}$ enforces it by default for the messages labeled $\texttt{[1...nsrc]}$. See next section for usage examples. 

\subsection{\textcolor{Chapter }{provess}}
\logpage{[ 3, 1, 3 ]}\nobreak
\hyperdef{L}{X7D75794B7EAD6A00}{}
{\noindent\textcolor{FuncColor}{$\triangleright$\ \ \texttt{provess({\mdseries\slshape Asets, nvars, svec, F, optargs})\index{provess@\texttt{provess}}
\label{provess}
}\hfill{\scriptsize (function)}}\\
\textbf{\indent Returns:\ }
A list 



 This function checks if there is a multi-linear secret sharing scheme with
secret and share sizes given by $\texttt{svec}$ and the access structure $\texttt{Asets}$ with one dealer and $\texttt{nvars-1}$ parties. It accepts following arguments: 
\begin{itemize}
\item  $\texttt{Asets}$ - A list of authorized sets each specified as a subset of $[\texttt{nvars}-1]$ 
\item  $\texttt{nvars}$ - Number of participants (including one dealer) 
\item  $\texttt{svec}$ - vector of integer share sizes understood as number of $\mathbb{F}_q$ symbols, with index 1 giving secret size and index $i,i\in \{2,...,\texttt{nvars}\}$ specifying share sizes of different parties 
\item  $\texttt{F}$ - The finite field over which we wish to find a multi-linear scheme. It can be
defined by invoking the following in GAP: 
\end{itemize}
 
\begin{Verbatim}[fontsize=\small,frame=single,label=Code]
  q:=4;;
  F:= GF(q);; # here q must be a prime power
\end{Verbatim}
 
\begin{itemize}
\item  $\texttt{optargs}$ is a list of optional arguments $\texttt{[lazy,..]}$ where 
\begin{itemize}
\item  $\texttt{lazy}$ disables lazy evaluation of transporter maps if set to $\texttt{false}$, which is enabled by default in GAP. 
\end{itemize}
 
\end{itemize}
 Returns a list $\texttt{[true,code]}$ if there exists such a scheme and $\texttt{code}$ is the vector linear code over $GF(q)$ Returns a list $\texttt{[false]}$ otherwise, indicating that no such scheme exists }

 $\textbf{NOTE:}$ No input checking is performed to verify if input $\texttt{Asets}$ follows labeling convensions. The map returned with a linear scheme is a map $f:[\texttt{nvars}]\rightarrow [\texttt{nvars}]$ with dealer associated with label 1 and parties associated with labels $\{2,...,\texttt{nvars}\}$. See next section for usage examples. 

\subsection{\textcolor{Chapter }{DisplayCode}}
\logpage{[ 3, 1, 4 ]}\nobreak
\hyperdef{L}{X83804D1E7BC6B451}{}
{\noindent\textcolor{FuncColor}{$\triangleright$\ \ \texttt{DisplayCode({\mdseries\slshape code})\index{DisplayCode@\texttt{DisplayCode}}
\label{DisplayCode}
}\hfill{\scriptsize (function)}}\\
\textbf{\indent Returns:\ }
nothing 



 Displays a network code (or an integer polymatroid). It accepts 1 argument: 
\begin{itemize}
\item  $\texttt{code}$ - A list $\texttt{[s,map]}$ containing 2 elements: 
\begin{itemize}
\item  $\texttt{s}$ - A list of subspaces where is subspace is itself a list of basis vectors 
\item  $\texttt{map}$ - A GAP record mapping subspaces in $\texttt{s}$ to network messages or to polymatroid ground set elements 
\end{itemize}
 
\end{itemize}
 Returns nothing }

 
\begin{Verbatim}[commandchars=!@|,fontsize=\small,frame=single,label=Example]
  !gapprompt@gap>| !gapinput@s:=[ [ [ 0*Z(2), 0*Z(2), Z(2)^0 ] ], [ [ 0*Z(2), Z(2)^0, 0*Z(2) ] ],\|
  !gapprompt@>| !gapinput@[ [ 0*Z(2), Z(2)^0, Z(2)^0 ] ], [ [ Z(2)^0, 0*Z(2), 0*Z(2) ] ],\|
  !gapprompt@>| !gapinput@[ [ Z(2)^0, 0*Z(2), Z(2)^0 ] ], [ [ Z(2)^0, Z(2)^0, 0*Z(2) ] ],\|
  !gapprompt@>| !gapinput@[ [ Z(2)^0, Z(2)^0, Z(2)^0 ] ] ];;|
  !gapprompt@gap>| !gapinput@map:=rec( 1 := 1, 2 := 2, 3 := 4, 4 := 3, 5 := 6, 6 := 5, 7 := 7 );;|
  !gapprompt@gap>| !gapinput@DisplayCode([s,map]);;|
  1->1
   . . 1
  =============================
  2->2
   . 1 .
  =============================
  3->4
   . 1 1
  =============================
  4->3
   1 . .
  =============================
  5->6
   1 . 1
  =============================
  6->5
  1 1 .
  =============================
  7->7
   1 1 1
  =============================
\end{Verbatim}
 }

 
\section{\textcolor{Chapter }{A catalog of examples}}\label{Chapter_Usage_Section_A_catalog_of_examples}
\logpage{[ 3, 2, 0 ]}
\hyperdef{L}{X838C78867A52DDA3}{}
{
  $\texttt{itap}$ comes equipped with a catalog of examples, which contains well-known network
coding instances and integer polymatroids. This catalog is intended to be of
help to the user for getting started with using $\texttt{itap}$. Most of the network coding instances in this catalog can be found in \cite{YeungBook} and \cite{DFZMatroidNetworks}. Some of the integer polymatroids in the catalog are taken from \href{http://code.ucsd.edu/zeger/linrank/} {\texttt{http://code.ucsd.edu/zeger/linrank/}}. These polymatroids correspond to the extreme rays of the cone of linear rank
inequalities in 5 variables, found by Dougherty, Freiling and Zeger. See \cite{DFZ2009Ineqfor5var} for details. 

\subsection{\textcolor{Chapter }{FanoNet}}
\logpage{[ 3, 2, 1 ]}\nobreak
\hyperdef{L}{X7C8C3B34834C6C8F}{}
{\noindent\textcolor{FuncColor}{$\triangleright$\ \ \texttt{FanoNet({\mdseries\slshape })\index{FanoNet@\texttt{FanoNet}}
\label{FanoNet}
}\hfill{\scriptsize (function)}}\\
\textbf{\indent Returns:\ }
A list 



 Returns the Fano instance. It accepts no arguments. Returns a list. }

 
\begin{Verbatim}[commandchars=!@|,fontsize=\small,frame=single,label=Example]
  !gapprompt@gap>| !gapinput@FanoNet();|
  [ [ [ [ 1, 2 ], [ 1, 2, 4 ] ], [ [ 2, 3 ], [ 2, 3, 5 ] ],
       [ [ 4, 5 ], [ 4, 5, 6 ] ], [ [ 3, 4 ], [ 3, 4, 7 ] ],
       [ [ 1, 6 ], [ 3, 1, 6 ] ], [ [ 6, 7 ], [ 2, 6, 7 ] ],
       [ [ 5, 7 ], [ 1, 5, 7 ] ] ], 3, 7 ]
  !gapprompt@gap>| !gapinput@rlist:=proverate(FanoNet(),[1,1,1,1,1,1,1],GF(2),[]);;|
  !gapprompt@gap>| !gapinput@rlist[1]; # Fano matroid is representable over GF(2)|
  true
  !gapprompt@gap>| !gapinput@DisplayCode(rlist[2]);|
  1->1
   . . 1
  =============================
  2->2
   . 1 .
  =============================
  3->4
   . 1 1
  =============================
  4->3
   1 . .
  =============================
  5->6
   1 . 1
  =============================
  6->5
   1 1 .
  =============================
  7->7
   1 1 1
  =============================
  !gapprompt@gap>| !gapinput@rlist:=proverate(FanoNet(),[1,1,1,1,1,1,1],GF(3),[]);;|
  !gapprompt@gap>| !gapinput@rlist[1]; # Fano matroid is not representable over GF(3)|
  false
\end{Verbatim}
 

\subsection{\textcolor{Chapter }{NonFanoNet}}
\logpage{[ 3, 2, 2 ]}\nobreak
\hyperdef{L}{X7FFA7A277C61E076}{}
{\noindent\textcolor{FuncColor}{$\triangleright$\ \ \texttt{NonFanoNet({\mdseries\slshape })\index{NonFanoNet@\texttt{NonFanoNet}}
\label{NonFanoNet}
}\hfill{\scriptsize (function)}}\\
\textbf{\indent Returns:\ }
A list 



 Returns the NonFano instance. It accepts no arguments. Returns a list. }

 
\begin{Verbatim}[commandchars=!@|,fontsize=\small,frame=single,label=Example]
  !gapprompt@gap>| !gapinput@NonFanoNet();|
  !gapprompt@gap>| !gapinput@gap> NonFanoNet();|
  [ [ [ [ 1, 2, 3 ], [ 1, 2, 3, 4 ] ], [ [ 1, 2 ], [ 1, 2, 5 ] ],
        [ [ 1, 3 ], [ 1, 3, 6 ] ], [ [ 2, 3 ], [ 2, 3, 7 ] ],
        [ [ 4, 5 ], [ 3, 4, 5 ] ], [ [ 4, 6 ], [ 2, 4, 6 ] ],
        [ [ 4, 7 ], [ 1, 4, 7 ] ] ], 3, 7 ]
\end{Verbatim}
 

\subsection{\textcolor{Chapter }{VamosNet}}
\logpage{[ 3, 2, 3 ]}\nobreak
\hyperdef{L}{X826AC373821A9EE2}{}
{\noindent\textcolor{FuncColor}{$\triangleright$\ \ \texttt{VamosNet({\mdseries\slshape })\index{VamosNet@\texttt{VamosNet}}
\label{VamosNet}
}\hfill{\scriptsize (function)}}\\
\textbf{\indent Returns:\ }
A list 



 Returns the Vamos instance. It accepts no arguments. Returns a list. }

 
\begin{Verbatim}[commandchars=!@|,fontsize=\small,frame=single,label=Example]
  !gapprompt@gap>| !gapinput@VamosNet();|
  [ [ [ [ 1, 2, 3, 4 ], [ 1, 2, 3, 4, 5 ] ],
        [ [ 1, 2, 5 ], [ 1, 2, 5, 6 ] ],
        [ [ 2, 3, 6 ], [ 2, 3, 6, 7 ] ],
        [ [ 3, 4, 7 ], [ 3, 4, 7, 8 ] ],
        [ [ 4, 8 ], [ 2, 4, 8 ] ],
        [ [ 2, 3, 4, 8 ], [ 1, 2, 3, 4, 8 ] ],
        [ [ 1, 4, 5, 8 ], [ 1, 2, 3, 4, 5, 8 ] ],
        [ [ 1, 2, 3, 7 ], [ 1, 2, 3, 4, 7 ] ],
        [ [ 1, 5, 7 ], [ 1, 3, 5, 7 ] ] ], 3, 7 ]
\end{Verbatim}
 

\subsection{\textcolor{Chapter }{U2kNet}}
\logpage{[ 3, 2, 4 ]}\nobreak
\hyperdef{L}{X8514B6657A3CC6FF}{}
{\noindent\textcolor{FuncColor}{$\triangleright$\ \ \texttt{U2kNet({\mdseries\slshape })\index{U2kNet@\texttt{U2kNet}}
\label{U2kNet}
}\hfill{\scriptsize (function)}}\\
\textbf{\indent Returns:\ }
A list 



 Returns the instance associated with uniform matroid $U^2_k$. It accepts one argument $\texttt{k}$ specifying the size of uniform matroid. Returns a list. }

 
\begin{Verbatim}[commandchars=!@|,fontsize=\small,frame=single,label=Example]
  !gapprompt@gap>| !gapinput@U2kNet(4);|
  [ [ [ [ 1, 2 ], [ 1, 2, 3 ] ], [ [ 1, 2 ], [ 1, 2, 4 ] ],
        [ [ 1, 3 ], [ 1, 2, 3 ] ], [ [ 1, 3 ], [ 1, 3, 4 ] ],
        [ [ 1, 4 ], [ 1, 2, 4 ] ], [ [ 1, 4 ], [ 1, 3, 4 ] ],
        [ [ 2, 3 ], [ 1, 2, 3 ] ], [ [ 2, 3 ], [ 2, 3, 4 ] ],
        [ [ 2, 4 ], [ 1, 2, 4 ] ], [ [ 2, 4 ], [ 2, 3, 4 ] ],
        [ [ 3, 4 ], [ 1, 3, 4 ] ], [ [ 3, 4 ], [ 2, 3, 4 ] ]
       ], 2, 4 ]
\end{Verbatim}
 

\subsection{\textcolor{Chapter }{ButterflyNet}}
\logpage{[ 3, 2, 5 ]}\nobreak
\hyperdef{L}{X8329D29E7FAC98B3}{}
{\noindent\textcolor{FuncColor}{$\triangleright$\ \ \texttt{ButterflyNet({\mdseries\slshape })\index{ButterflyNet@\texttt{ButterflyNet}}
\label{ButterflyNet}
}\hfill{\scriptsize (function)}}\\
\textbf{\indent Returns:\ }
A list 



 Returns the Butterfly instance. It accepts no arguments. Returns a list. }

 
\begin{Verbatim}[commandchars=!@|,fontsize=\small,frame=single,label=Example]
  !gapprompt@gap>| !gapinput@ButterflyNet();|
  [ [ [ [ 1, 2 ], [ 1, 2, 3 ] ], [ [ 1, 3 ], [ 1, 2, 3 ] ],
       [ [ 2, 3 ], [ 1, 2, 3 ] ] ], 2, 3 ]
  !gapprompt@gap>| !gapinput@U2kNet(3)=ButterflyNet();|
  true
\end{Verbatim}
 

\subsection{\textcolor{Chapter }{Subspace5}}
\logpage{[ 3, 2, 6 ]}\nobreak
\hyperdef{L}{X808E1A5081F14455}{}
{\noindent\textcolor{FuncColor}{$\triangleright$\ \ \texttt{Subspace5({\mdseries\slshape })\index{Subspace5@\texttt{Subspace5}}
\label{Subspace5}
}\hfill{\scriptsize (function)}}\\
\textbf{\indent Returns:\ }
A list 



 Returns the extreme rays of cone formed by linear rank inequalities in 5
variables. It accepts no arguments. Returns a list. }

 
\begin{Verbatim}[commandchars=!@|,fontsize=\small,frame=single,label=Example]
  !gapprompt@gap>| !gapinput@rays5:=Subspace5();;|
  !gapprompt@gap>| !gapinput@Size(rays5);|
  162
  !gapprompt@gap>| !gapinput@rlist:=proverep(rays5[46],5,GF(2),[])|
  !gapprompt@>| !gapinput@rlist[1];|
  true
  !gapprompt@gap>| !gapinput@gap> DisplayCode(rlist[2]);|
  1->4
   . . . 1
  =============================
  2->5
   . . 1 .
  =============================
  3->3
   . 1 . .
  =============================
  4->2
   1 . . .
   . . 1 1
  =============================
  5->1
   1 . . 1
   . 1 1 1
  =============================
\end{Verbatim}
 

\subsection{\textcolor{Chapter }{BenaloahLeichter}}
\logpage{[ 3, 2, 7 ]}\nobreak
\hyperdef{L}{X7AEEE54F7E0FD055}{}
{\noindent\textcolor{FuncColor}{$\triangleright$\ \ \texttt{BenaloahLeichter({\mdseries\slshape })\index{BenaloahLeichter@\texttt{BenaloahLeichter}}
\label{BenaloahLeichter}
}\hfill{\scriptsize (function)}}\\
\textbf{\indent Returns:\ }
A list of lists specifing authorized subsets of $\{1,2,3,4\}$ 



 Returns the access structure associated with secret sharing scheme of Benaloah
and Leichter that was used to show that share sizes can be larger than secret
size. See \cite{Benaloh90} for details. It accepts no arguments. Returns a list. }

 
\begin{Verbatim}[commandchars=!@|,fontsize=\small,frame=single,label=Example]
  !gapprompt@gap>| !gapinput@B:=BenaloahLeichter();|
  [ [ 1, 2 ], [ 2, 3 ], [ 3, 4 ] ]
  !gapprompt@gap>| !gapinput@rlist:=provess(B,5,[2,2,3,3,2],GF(2),[]);;|
  !gapprompt@gap>| !gapinput@rlist[1];|
  true
  !gapprompt@gap>| !gapinput@DisplayCode(rlist[2]);|
  1->1
   . . . . 1 .
  . . . . . 1
  =============================
  2->2
  . . 1 . . .
  . . . 1 . .
  =============================
  3->3
  . 1 . . . .
  . . 1 . . 1
  . . . 1 1 .
  =============================
  4->5
  1 . . . . .
  . 1 . . . .
  =============================
  5->4
  1 . . . . 1
  . 1 . . 1 .
  . . 1 . . .
  =============================
\end{Verbatim}
 

\subsection{\textcolor{Chapter }{Threshold}}
\logpage{[ 3, 2, 8 ]}\nobreak
\hyperdef{L}{X7C7738767CA2118A}{}
{\noindent\textcolor{FuncColor}{$\triangleright$\ \ \texttt{Threshold({\mdseries\slshape })\index{Threshold@\texttt{Threshold}}
\label{Threshold}
}\hfill{\scriptsize (function)}}\\
\textbf{\indent Returns:\ }
A list of lists specifing authorized subsets of $[n]]$ 



 Returns the access structure associated with Shamir's $(k,n)$ threshold scheme. See \cite{Shamirhowto79} for details. It accepts following arguments: 
\begin{itemize}
\item  $\texttt{n}$ - number of shares 
\item  $\texttt{k}$ - threshold 
\end{itemize}
 }

 
\begin{Verbatim}[commandchars=!@|,fontsize=\small,frame=single,label=Example]
  !gapprompt@gap>| !gapinput@T:=Threshold(4,2);|
  [ [ 1, 2 ], [ 1, 3 ], [ 1, 4 ], [ 2, 3 ], [ 2, 4 ], [ 3, 4 ] ]
  !gapprompt@gap>| !gapinput@rlist:=provess(T,5,[1,1,1,1,1],GF(2),[]);|
  [ false ]
  !gapprompt@gap>| !gapinput@rlist:=provess(T,5,[1,1,1,1,1],GF(3),[]);|
  [ false ]
  !gapprompt@gap>| !gapinput@rlist:=provess(T,5,[1,1,1,1,1],GF(5),[]);;|
  !gapprompt@gap>| !gapinput@rlist[1];|
  true
  !gapprompt@gap>| !gapinput@DisplayCode(rlist[2]);|
  1->1
   . 1
  =============================
  2->2
   1 .
  =============================
  3->3
   1 1
  =============================
  4->4
   1 2
  =============================
  5->5
  1 4
  =============================
\end{Verbatim}
 }

 }

 \def\bibname{References\logpage{[ "Bib", 0, 0 ]}
\hyperdef{L}{X7A6F98FD85F02BFE}{}
}

\bibliographystyle{alpha}
\bibliography{itap.bib}

\addcontentsline{toc}{chapter}{References}

\def\indexname{Index\logpage{[ "Ind", 0, 0 ]}
\hyperdef{L}{X83A0356F839C696F}{}
}

\cleardoublepage
\phantomsection
\addcontentsline{toc}{chapter}{Index}


\printindex

\newpage
\immediate\write\pagenrlog{["End"], \arabic{page}];}
\immediate\closeout\pagenrlog
\end{document}
